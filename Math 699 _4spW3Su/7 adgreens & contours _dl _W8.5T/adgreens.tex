\documentclass[notitlepage,12pt]{article}

\pdfoutput=1

\usepackage{amsmath}
\usepackage{amssymb,amsfonts}
\usepackage{times}
\usepackage{graphicx}
\usepackage{bm}
\usepackage{subfigure}
\usepackage{color}
\usepackage[numbers]{natbib}
\usepackage{hyperref}
\usepackage[letterpaper]{geometry}
\usepackage[capitalize]{cleveref}

\graphicspath{{figs/}}

% Commands
\newcommand{\ie}{\textit{i.e.}}
\newcommand{\etal}{\textit{et~al.}}
\newcommand{\mathnotation}[2]{\newcommand{#1}{\ensuremath{#2}}}

\newcommand{\Order}[1]{\mathrm{O}(#1)}      % Higher-order terms
\newcommand{\order}[1]{\mathrm{o}(#1)}      % Higher-order terms

\newcommand{\jlt}[1]{{\color{red}{(*** #1)}}}
%\newcommand{\jlt}[1]{}

%
% Symbols
%

\renewcommand{\l}{\left}
\renewcommand{\r}{\right}
\mathnotation{\pd}{\partial}
\mathnotation{\ldef}{\mathrel{\raisebox{.069ex}{:}\!\!=}}% Left define
\mathnotation{\rdef}{\mathrel{=\!\!\raisebox{.069ex}{:}}}% Right define
\mathnotation{\dint}{\mathop{}\!\mathrm{d}} % Differential, in an integral
\mathnotation{\ee}{\mathrm{e}}              % Base of natural log
\mathnotation{\imi}{\mathrm{i}}             % Imaginary i


\begin{document}

\title{Green's function for the \\ advection-diffusion equation}

%\author{Jean-Luc Thiffeault%
%  \thanks{Department of Mathematics, University of Wisconsin --
%  Madison, 480 Lincoln Dr., Madison, WI 53706, USA.
%  E-mail: \href{mailto:jeanluc@math.wisc.edu}{jeanluc@math.wisc.edu}.}}

\date{\today}

\maketitle

\mathnotation{\xc}{x}
\mathnotation{\yc}{y}
\mathnotation{\rc}{r}
\mathnotation{\rv}{\bm{r}}
\mathnotation{\lapl}{\Delta}
\mathnotation{\kc}{k}
\mathnotation{\kv}{\bm{\kc}}
\mathnotation{\ac}{a}

\section{Green's function from Bessel's equation}

\begin{equation}
  U\,\pd_\yc \phi = D\,\lapl\phi + \delta(\rv)
\end{equation}
Let $\psi(\rv) = \ee^{-U\yc/2D}\phi(\rv)$; then $\psi$ satisfies
\begin{equation}
  -D\,\lapl\psi + \frac{U^2}{4D}\,\psi = \delta(\rv).
  \label{eq:Helm0}
\end{equation}
This is a Hemholtz equation.  For convenience we define the inverse length
scale~$\ac = U/2D$:
\begin{equation}
  -\lapl\psi + \ac^2\,\psi = \delta(\rv)/D.
  \label{eq:Helm}
\end{equation}
The solution to this must be dependent on~$\rc$ only, so we try to solve:
\begin{equation}
  \frac{1}{\rc}\frac{d}{d\rc}\l(\rc\frac{\pd\psi}{\pd\rc}\r)
  -
  \ac^2\psi = 0.
\end{equation}
This is a modified Bessel equation; the solution that decays
as~$\rc\rightarrow\infty$ is
\begin{equation}
  \psi(\rc) = c_1\,K_0(\ac\rc).
\end{equation}
The Green's function we seek is then of the form
\begin{equation}
  \phi(\rv)
  =
  \ee^{\ac\yc}\,\psi(\rc)
  =
  c_1\,\ee^{\ac\yc}\,K_0(\ac\rc).
\end{equation}
To find the constant~$\ac$, we observe that for small~$x$
\begin{equation}
  K_0(x) \sim -\log x,
  \qquad
  x \rightarrow 0.
\end{equation}
Hence, for the Green's function to limit to that for the Laplacian with a
point source of strength~$-1/D$ as~$\rc\rightarrow0$, we require
\begin{equation}
  \phi(\rv)
  \sim
  -c_1\,\log\rc
  \sim
  -\frac{1}{2\pi D}\log\rc,
  \qquad
  \rc \rightarrow 0.
\end{equation}
We obtain finally the Green's function
\begin{equation}
  \phi(\rv)
  =
  \frac{1}{2\pi D}\,\ee^{\ac\yc}\,K_0(\ac\rc),
  \qquad
  \ac = \frac{U}{2D}.
  \label{eq:GF}
\end{equation}
A useful asymptotic form for this is
\begin{equation}
  \phi(\rv) \sim \frac{\ee^{\ac(\yc - \rc)}}{\sqrt{4\pi D U \rc}},
  \qquad
  \rc \rightarrow\infty.
  \label{eq:GFasym}
\end{equation}
In particular, this has very different limits along the~$\xc=0$ axis dependent
on whether we are upstream ($\yc \rightarrow -\infty$) or downstream
($\yc \rightarrow \infty$):
\begin{equation}
  \phi \sim \frac{1}{\sqrt{4\pi D U \lvert y\rvert}}
  \begin{cases}
    1\,, \qquad &y \rightarrow \infty; \\
    \ee^{-2\ac|\yc|}\,, \qquad &y \rightarrow -\infty.
  \end{cases}
\end{equation}


\section{The shape of contours}

\mathnotation{\cc}{c}
\mathnotation{\ycmin}{\yc_{\mathrm{min}}}
\mathnotation{\ycmax}{\yc_{\mathrm{max}}}

To find the shape of large contours~$\phi(\rv)=\cc/\sqrt{4\pi D U} > 0$ we use
the asymptotic form \cref{eq:GFasym}:
\begin{equation}
  \ee^{\ac(\yc - \rc)} = \cc\,\sqrt{\rc}.
  \label{eq:contourasym}
\end{equation}
The furthest extent of the contour for~$\yc > 0$ is obtained by
setting~$\rc = \yc = \ycmax$ and then solving
\begin{equation}
  1 = \cc\,\sqrt{\ycmax}
  \quad\Longleftrightarrow\quad
  \ycmax = 1/\cc^2.
\end{equation}
We let furthest extent of the countour for~$\yc<0$ be~$\yc=-\yc_{\text{min}}$.
Assuming that~$\yc_{\text{min}}$ is large and that \cref{eq:GFasym} is still
valid, we set~$\rc = -\yc = \ycmin$ and then solve
%\begin{equation}
%  \ee^{-\ac\ycmin} = \cc\,\sqrt{\ycmin}.
%\end{equation}
\begin{equation}
  \ee^{-4\ac\ycmin} = \cc^2\,\ycmin.
\end{equation}
The solution to this is given in terms of the
\href{http://mathworld.wolfram.com/LambertW-Function.html}{Lambert
  $W$-function} as
\begin{equation}
  \ycmin = \frac{1}{4\ac}\,W(4\ac/\cc^2).
\end{equation}
For small~$\cc$, this has asymptotic expansion
\begin{equation}
  \ycmin \sim \frac{1}{4\ac}
  \l(
  \log(4\ac/\cc^2) - \log\log(4\ac/\cc^2)
  \r),
  \qquad
  \cc \rightarrow 0.
\end{equation}
This is indeed getting slowly larger as~$\cc\rightarrow0$, consistent with
using the approximation~\cref{eq:GFasym}, but the extent of the contour
in~$\yc$ is completely dominated by~$\ycmax$.  We can thus safely
use~$\ycmin=0$ when computing the area.

Now that we know the limits in~$\yc$, to compute the area as a function
of~$\yc$ we use \cref{eq:contourasym} 
\begin{equation}
  \frac{\ee^{\ac(\yc - \sqrt{\xc^2 + \yc^2})}}{(\xc^2 + \yc^2)^{1/4}} = \cc
  \label{eq:contourasymxy}
\end{equation}
and try to solve for~$\xc$.  We rescale $X = \sqrt{\ac}\,\cc\,\xc$,
$Y = \cc^2\yc$, and then Taylor expand:
\begin{equation}
  \frac{\ee^{-X^2/2Y}}{\sqrt{Y}} + \Order{\cc^2} = 1.
\end{equation}
%\begin{equation}
%  \ee^{-X^2/2Y} = \sqrt{Y}\,(1 + \Order{\cc^2}).
%\end{equation}
%\begin{equation}
%  -X^2/2Y = \log\sqrt{Y} + \log(1 + \Order{\cc^2}).
%\end{equation}
%\begin{equation}
%  X^2 = -Y\log Y + \Order{\cc^2}.
%\end{equation}
%\begin{equation}
%  X = \sqrt{-Y\log Y} + \Order{\cc^2}.
%\end{equation}
Solving for $X$, we find for the shape of the contour
\begin{equation}
  X = \sqrt{Y\log Y^{-1}} + \Order{\cc^2},
  \qquad
  0 < Y < 1,
\end{equation}
or in terms of the unscaled variables,
\begin{equation}
  \xc_{\mathrm{max}}(\yc) = \sqrt{(\yc/\ac)\log(\cc^2\yc)^{-1}}
  + \Order{\cc},
  \qquad
  0 < \yc < \cc^{-2}.
  \label{eq:xapprox}
\end{equation}
Now we can compute the estimated area as
\begin{equation}
  A
  =
  \frac{2\cc^{-3}}{\sqrt\ac}\int_0^1 X(Y)\dint Y
  =
  \frac{2}{3}\,\sqrt{\frac{2\pi}{3\ac}}\,\cc^{-3}.
  \label{eq:A}
\end{equation}
The crucial observation is that~$A \sim \cc^{-3}$.  Hence,
\begin{equation}
  \ycmax \sim \cc^{-2} \sim A^{2/3}
\end{equation}
as we find in the simulations.  Figure~\ref{fig:contours} compares these
approximate contours to numerical simulations.

\begin{figure}
  \begin{center}
    \includegraphics[width=.65\textwidth]{contours}
  \end{center}
  \caption{Numerical contours (solid) compared to the
    approximation~\eqref{eq:xapprox}.}
  \label{fig:contours}
\end{figure}

\jlt{Note that for us the area is twice the number of particles, since we have
a checkerboard pattern.  Also, I think~$\ac=1$ is the right scaling,
since~$UT=1$ gridpoint, and $\sqrt{2DT}=1$ gridpoint as well.}



\clearpage

\section{Using Fourier transform}

\jlt{This is not necessary.}

We take the Fourier transform of \cref{eq:Helm}:
\begin{equation}
  \kc^2\,\hat\psi + \ac^2\,\hat\psi = 1/D.
  \label{eq:Helmk}
\end{equation}
Here we define the forward transform
\begin{equation}
  \hat\psi(\kv) = \int \psi(\rv)\, \ee^{-\imi\kv\cdot\rv} \dint \rv
\end{equation}
and the inverse transform
\begin{equation}
  \psi(\rv) = \frac{1}{2\pi}
  \int \hat\psi(\kv)\, \ee^{\imi\kv\cdot\rv} \dint \kv.
\end{equation}
Solving \cref{eq:Helmk} for~$\hat\psi$ and inverting, we have
\begin{equation}
  \psi(\rv)
  =
  \frac{1}{2\pi D}
  \int
  \frac{\ee^{\imi\kv\cdot\rv}}{\kc^2 + \ac^2} \dint \kv.
\end{equation}
We can transform the~$\kv$ integral to polar coordinates
with~$\kv = \kc\,(\cos\varphi,\sin\varphi)$
and~$\rv = \rc\,(\cos\theta,\sin\theta)$:
\begin{equation}
  \psi(\rv)
  =
  \frac{1}{2\pi D}
  \int_0^{2\pi}
  \int_0^\infty
  \frac{\ee^{\imi\kc\rc\cos(\varphi-\theta)}}{\kc^2 + \ac^2}
  \,\kc\dint \kc\dint \varphi.
\end{equation}
Because of periodicity, we can translate the~$\varphi$ integral by~$\theta$
and obtain the $\theta$-independent form
\begin{equation}
  \psi(\rc)
  =
  \frac{1}{2\pi D}
  \int_0^{2\pi}
  \int_0^\infty
  \frac{\ee^{\imi\kc\rc\cos\varphi}}{\kc^2 + \ac^2}
  \,\kc\dint \kc\dint \varphi.
\end{equation}
Mathematica can carry this one out directly to give
\begin{equation}
  \psi(\rc)
  =
  \frac{1}{D}
  \,
  K_0(\ac\rc).
\end{equation}



%\bibliographystyle{jlt}
%\bibliography{journals_abbrev,articles}

\end{document}
