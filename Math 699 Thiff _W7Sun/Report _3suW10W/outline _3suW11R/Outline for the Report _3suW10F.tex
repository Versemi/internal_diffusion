%%%%%%%%%%%%%%%%%%%%%%%%%%%%%%%%%%%%%%%%%
% 8/22 _3suW10W

\documentclass[letterpaper]{article}
\usepackage[paperwidth=8.5in, paperheight=11in]{geometry}
\linespread{1.2}
\normalsize

\usepackage[utf8]{inputenc}

\usepackage{indentfirst}
\usepackage[utf8]{inputenc}
\usepackage{geometry}

\usepackage{amsmath,amsfonts,amsthm} % Math packages

\usepackage[english]{babel}
\usepackage[autostyle]{csquotes}

\numberwithin{equation}{section} % Number equations within sections (i.e. 1.1, 1.2, 2.1, 2.2 instead of 1, 2, 3, 4)
\numberwithin{figure}{section} % Number figures within sections (i.e. 1.1, 1.2, 2.1, 2.2 instead of 1, 2, 3, 4)
\numberwithin{table}{section} % Number tables within sections (i.e. 1.1, 1.2, 2.1, 2.2 instead of 1, 2, 3, 4)

\title{	
	\normalfont \normalsize 
	\huge The Convergence of IDLA to a Circle \\ % The assignment title
}

\author{Jean-Luc Thiffeault and Ruojun Wang} % Your name

\date{\normalsize\today} % Today's date or a custom date

\begin{document}
	
\maketitle % Print the title
	
	%----------------------------------------------------------------------------------------
	%	PROBLEM 1
	%----------------------------------------------------------------------------------------
	
(introduction?)

\section{An IDLA Simulation with N Particles}
	
Internal diffusion-limited aggregation (IDLA) is ... (definition of IDLA). 

MATLAB codes simulate an IDLA process as following: ...

From the graph generated by MATLAB codes, we observe that the boundary of the shape constructed by the IDLA process tends to be smooth and to imitate a circle, as the value of \texttt{Npart} (the number of particles) become larger. We want to understand to what extent the shape constructed by the process converges to a circle. (The numerical value of standard deviation ...) The error from the simulation, which is marked by $\sigma_{sim}$, consists of two errors, the geometric error $\sigma_{geom}$ and the statistical error $\sigma_{stat}$. 	

\section{The Geometric Error}

A discretized circle can be constructed by two different kinds of algorithms, in order to obtain the geometric error $\sigma_{geom}$. 

\subsection{The Numerical Standard Deviation}

\subsubsection{The First Algorithm (\enquote{Nonremove} Case)}
We consider a grid (a lattice of circle) in the plane with centers at coordinates $p=(m, n)\in\mathbb{Z}^2$. The center of circle drawn locates at the origin $(0,0)$. Take a circle of radius $R$, centered on the origin. A continuous discretization of the circle is an ordered set of distinct pixels which the boundary of the circle passes through

\begin{align} 
\mathcal{D}_R=(p_{i})_{0 \leq i \leq N-1} = (m_i, n_i)_{0 \leq i \leq N-1},
\end{align}

where $m_i$ denotes the horizontal coordinate of the center of a pixel and $n_i$ denotes the vertical one of that.

The algorithm can be realized by MATLAB codes as following: ... Given $R=10000$, $\sigma_{geom}=0.3729$.

	
\subsubsection{The Second Algorithm (\enquote{Remove} Case)}

Another ways to construct a discretized consisting of less pixels occupied than those in the first algorithm. The algorithm can be realized by MATLAB codes: ... When $R=10000$, $\sigma_{geom}=0.2624$, which is smaller than the value obtained from the first algorithm.

\subsection{The Upper Bound of $L_{2}$ Error of the Discretization}

(A derivation of $L_{2}$ Error of the discretization.)

\begin{align} 
\text{Err}_2 \ \mathcal{D}_R=(\frac{1}{N} \sum_{i=0}^{N-1} (m_i^2+n_i^2)-R^2)^{\frac{1}{2}}.
\end{align}


In both algorithm to construct a discretized circle, $\sigma_{geom}$ can be approximated by Err$_2 \ \mathcal{D}_R$ when $R \rightarrow \infty$, since ... Err$_2 \ \mathcal{D}_R$ can be considered as the distance between the center of each pixel $p_i$ and the boundary of the circle which passes through this pixel. We denote Err$_2 \ \mathcal{D}_R$ as $d$ in this case. By computation(...), if the small piece is randomly distributed, the expectation value $\mathbb{E}\ d^2 \approx 0.1667$. 

For both algorithms, a analytical upper bound can be found for $d^2$, which approximately equal to $\sigma_{geom}^2$ when $R \rightarrow \infty$. $\sigma_{geom}^2 \leq \frac{1}{2} \approx 0.5$ and then $\sigma_{geom} \leq 0.7071$.

(We sort pixels into 4 different types as following to lower the upper bound of $d$ ...)

\subsubsection{Sort Pixels into 4 Different Types}
According to the ways that the boundary of the circle passes through each pixel, we can sort $N$ pixels into 4 different types for the upper right quarter of the circle (other three quarters would just be the mirror images of the upper right one with respect to different axises of symmetry). As the algorithm provided above, the center of each pixel is represented by $(m_i, n_i)$. (graph?) The boundary of the circle passing by would have 2 different intersections with a pixel $p_i$. Hence, the four different type of grids can be given by the inequalities restricting the coordinates $(x,y)$ of the intersections. The first kind of pixel is provided by  

\begin{align} 
\text{one intersection: } x=m_i-\frac{1}{2}; \ n_i-\frac{1}{2}\leq y \leq n_i-\frac{1}{2}, \ \text{where} \ y=\sqrt{R^2-(m_i-\frac{1}{2})^2};
\end{align}

... (other inequalities)

(A deviation to find upper bound by evaluating these four types of pixels.) 

\begin{align} 
\sigma_{geom}^2 \leq \frac{1}{2} \approx 0.5
\end{align}

Also, $\sigma_{geom} \leq 0.7071$. (The upper bound is still the same. We need a further improvement.)

\subsubsection{An Observation of the Relation between the Fraction of Each Type of Pixels and Multiples of $R$}

(By derivation, $(1-\frac{1}{\sqrt{2}})R \approx 0.29289 R$ and $\frac{1}{\sqrt{2}}R \approx 0.70711 R$)

\paragraph{The \enquote{Nonremove} Case.}
Given by MATLAB codes, the numerical fraction of the four different type of pixels are 

\begin{align} 
\text{fraction of type 1}=0.29291; \\
\text{fraction of type 2}=0.29286; \\
\text{fraction of type 3}=0.20711; \\
\text{fraction of type 4}=0.20711; 
\end{align}

(By observation, the value in (2.5) and $(1-\frac{1}{\sqrt{2}})$ are close. A derivation to find the relation between these two. The upper bound of $d^2$ can be determined then: $d^2 \leq p_1 \ (\frac{1}{\sqrt{2}})^2 + p_2 \ (\frac{1}{2})^2 $.)

\paragraph{The \enquote{Remove} Case.}
	
Given by MATLAB codes, the numerical fraction of the four different type of pixels (for $R=10000$) are 

\begin{align} 
\text{fraction of type 1}=...; \\
\text{fraction of type 2}=...; \\
\text{fraction of type 3}=0.29291; \\
\text{fraction of type 4}=0.29291; 
\end{align}

By observation, the values in (2.11), (2.12), and $(1-\frac{1}{\sqrt{2}})$ are close. (However, to sort pixels into 4 types cannot help construct a relation between the fraction of each type of pixels and multiples of $R$ as we did for the \enquote{nonremove} case. We need to sort pixels into 6 different types to obtain an analytical solution.)

\subsubsection{Sort Pixels into 6 Different Types}

(A derivation to obtain $d^2 \leq p_1 \ (\frac{1}{\sqrt{2}})^2 + p_2 \ (\frac{1}{2})^2 $.)

(conclusion?)

(reference?)

\end{document}